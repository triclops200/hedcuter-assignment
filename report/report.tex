\documentclass[11pt]{article}

\usepackage{latexsym}
\usepackage{amsmath}
\usepackage{amssymb}
\usepackage{amsthm}
\usepackage{graphicx}
\usepackage{wrapfig}
\usepackage{pseudocode}
\usepackage{url}
\usepackage[backref, colorlinks=true, citecolor=red, urlcolor=blue, pdfauthor={Jyh-Ming Lien}]{hyperref}
\usepackage{svg}


\newcommand{\handout}[5]{
  \noindent
  \begin{center}
    \framebox{
      \vbox{
        \hbox to 5.78in { {\bf } \hfill #2 }
        \vspace{4mm}
        \hbox to 5.78in { {\Large \hfill #5  \hfill} }
        \vspace{2mm}
        \hbox to 5.78in { {\em #3 \hfill #4} }
      }
    }
  \end{center}
  \vspace*{4mm}
}

\newcommand{\lecture}[4]{\handout{#1}{#2}{#3}{#4}{#1}}

\newtheorem{theorem}{Theorem}
\newtheorem{corollary}[theorem]{Corollary}
\newtheorem{lemma}[theorem]{Lemma}
\newtheorem{observation}[theorem]{Observation}
\newtheorem{proposition}[theorem]{Proposition}
\newtheorem{definition}[theorem]{Definition}
\newtheorem{claim}[theorem]{Claim}
\newtheorem{fact}[theorem]{Fact}
\newtheorem{assumption}[theorem]{Assumption}

% 1-inch margins, from fullpage.sty by H.Partl, Version 2, Dec. 15, 1988.
\topmargin 0pt
\advance \topmargin by -\headheight
\advance \topmargin by -\headsep
\textheight 8.9in
\oddsidemargin 0pt
\evensidemargin \oddsidemargin
\marginparwidth 0.5in
\textwidth 6.5in

\parindent 0in
\parskip 1.5ex
% \renewcommand{\baselinestretch}{1.25}

\begin{document}

\lecture{Hedcuter Assignment}{Fall 2017}{Bryan Hoyle}{Computational Geometry}

\section{Summary of the two methods}

\subsection{hedcuter method}

This method uses the wavefront method to compute voronoi cells. First,
it oversizes the image in order to do subpixel sampling. Next, it
actually computes to which cell point to which each subpixel
belongs. To do this, it maintains a weighted queue of points keyed by
weighted distance from front (and if the weighted distances are the
same, it chooses priority based on the the horizontal posistions, and
if those are the same, the vertical positions). It then checks the
current subpixel it is on to see if any of its neighbors could belong
to a closer voronoi cell. If so, they are updated and the wavefront is
updated to contain the changed subpixel. Once it is done with all of
the wavefronts, it then does a reverse lookup for every pixel to add
it to the cell's ownership list. It then computes the weighted
centroid using the density function described
in~\cite{stippling}. Afterwards, it moves the center of each cell to
the respective centroid. It repeats this process until either the
desired number of iterations is reached or the displacement reached
some minimum. It then renders the stipples using average color of the
cell in RGB space and the radius is either uniform to what was set or,
if varying disk size is allowed, it's set to be scaled by the average
value of cell's color.

\subsection{voronoi method}

The voronoi method is rather easy. It iterates a weighted relaxation
based on the density function described in the paper. To compute the
voronoi diagram, it uses Fortune's algorithm. When it's computing the
centroid, it computes the clip lines of the cell and grabs a ``tile''
by using the bounding box of the cell. It then scales the tile to the
desired size (for use in subsampling). Then, it iterates over every
pixel in the tile and uses the clip lines to determine if the pixel is
in the cell. If it is in the cell, it updates both the weight of the
cell and the maximum possible weight of the cell by using the
intensity of the pixel. Then, it computes the radius of the stipple
(if it is set to bound by the cell, it uses the biggest possible
internal radius of the cell, otherwise, it uses the biggest possible
radius that has an edge intersecting the stipple). It uses the ratio
of the weight to the maximum possible weight to scale the radius of
the stipple (the darker the image in the voronoi cell, the larger the
radius). It then sets the center of the cell to the weighted
centroid. It repeats the loop of computing the voronoi diagram then
using weighted relaxation until the average displacement of the
relaxation is very small (meaning that it didn't change much). When it
is done, it just renders the diagram by sampling the cell point for
color (if color mode is enabled, otherwise, it uses a black circle)
and then rendering the stipples as circles of the computed radius.

\section{Comparison of the two methods}

\subsection{Do you get the same results by running the same program on the same image multiple times?}

The voronoi\_stippler program does give the same results each time,
however, the hedcuter executable does not. This is because the the
voronoi\_stippler code uses the same seed each time it is run for the
random number generator. The hedcuter code does not use the same seed
each time.

\begin{figure}[h!]
  \centering
  \begin{minipage}{.5\textwidth}
    \centering
    \includesvg[width=0.95\textwidth]{A1}
  \end{minipage}%
  \begin{minipage}{.5\textwidth}
    \centering
    \includesvg[width=0.95\textwidth]{A2}
  \end{minipage}
  \caption{Comparison of two runs of hedcuter on the same image using
    the same parameters. Notice the hair (circled in red) is different}
\end{figure}

\subsection{If you vary the number of the disks in the output images,
  do these implementations produce the same distribution in the final
  image? If not, why?}

It seems like the hedcuter implementation crowds the dark areas more
heavily (which makes sense, as it uses the darkness of a pixel when
calculating the voronoi diagram as part of the distance). Also, the
voronoi\_stippler seems more uniform in its final output.  See
figures~\ref{hc_b} and~\ref{vs_b} for comparison of output. A problem
with hedcuter arises here: with a disk size other than 1, it can
overcrowd the voronoi diagram and cause overlap, but it doesn't look
very good with a disk size of 1, as the disks are too small unless you
use a larg number os disks.

\begin{figure}[h!]
  \centering
  \begin{minipage}{.32\textwidth}
    \centering
    \includesvg[width=0.95\textwidth]{B1}
  \end{minipage}%
  \begin{minipage}{.32\textwidth}
    \centering
    \includesvg[width=0.95\textwidth]{B2}
  \end{minipage}
  \begin{minipage}{.32\textwidth}
    \centering
    \includesvg[width=0.95\textwidth]{B3}
  \end{minipage}
  \caption{Three runs of 100, 1000, and 10000 disks using the hedcuter
  program. Notice that it crowds more heavily towards black areas as
  well as has oversized disks which overlap.}\label{hc_b}
\end{figure}

\begin{figure}[h!]
  \centering
  \begin{minipage}{.32\textwidth}
    \centering
    \includesvg[width=0.95\textwidth]{B4}
  \end{minipage}%
  \begin{minipage}{.32\textwidth}
    \centering
    \includesvg[width=0.95\textwidth]{B5}
  \end{minipage}
  \begin{minipage}{.32\textwidth}
    \centering
    \includesvg[width=0.95\textwidth]{B6}
  \end{minipage}
  \caption{Three runs of 100, 1000, and 10000 disks using the
    voronoi\_stippler program. This output is more uniform across the
    image. Also, note how the implementation bounds the circles to
    stay within the voronoi cell, so there is no overlap.}\label{vs_b}
\end{figure}

\subsection{If you vary the number of the disks in the output images,
  is a method faster than the other?}

Yes. The voronoi\_stippler method actually speeds up with more
disks. Hedcuter gets slower the more disks were added. For example,
for figure~\ref{hc_b}, the times to compute were 32, 38, and 43
seconds respectively, however, for figure~\ref{vs_b}, the times to
compute were 54, 46, and 7 seconds, respectively.


\section{Improvement of hedcuter method}

\bibliographystyle{plain}
\bibliography{report}

\end{document}


