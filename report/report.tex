\documentclass[11pt]{article}

\usepackage{latexsym}
\usepackage{amsmath}
\usepackage{amssymb}
\usepackage{amsthm}
\usepackage{graphicx}
\usepackage{wrapfig}
\usepackage{pseudocode}
\usepackage{url}
\usepackage[backref, colorlinks=true, citecolor=red, urlcolor=blue, pdfauthor={Jyh-Ming Lien}]{hyperref}


\newcommand{\handout}[5]{
  \noindent
  \begin{center}
  \framebox{
    \vbox{
      \hbox to 5.78in { {\bf } \hfill #2 }
      \vspace{4mm}
      \hbox to 5.78in { {\Large \hfill #5  \hfill} }
      \vspace{2mm}
      \hbox to 5.78in { {\em #3 \hfill #4} }
    }
  }
  \end{center}
  \vspace*{4mm}
}

\newcommand{\lecture}[4]{\handout{#1}{#2}{#3}{#4}{#1}}

\newtheorem{theorem}{Theorem}
\newtheorem{corollary}[theorem]{Corollary}
\newtheorem{lemma}[theorem]{Lemma}
\newtheorem{observation}[theorem]{Observation}
\newtheorem{proposition}[theorem]{Proposition}
\newtheorem{definition}[theorem]{Definition}
\newtheorem{claim}[theorem]{Claim}
\newtheorem{fact}[theorem]{Fact}
\newtheorem{assumption}[theorem]{Assumption}

% 1-inch margins, from fullpage.sty by H.Partl, Version 2, Dec. 15, 1988.
\topmargin 0pt
\advance \topmargin by -\headheight
\advance \topmargin by -\headsep
\textheight 8.9in
\oddsidemargin 0pt
\evensidemargin \oddsidemargin
\marginparwidth 0.5in
\textwidth 6.5in

\parindent 0in
\parskip 1.5ex
%\renewcommand{\baselinestretch}{1.25}

\begin{document}

\lecture{Hedcuter Assignment}{Fall 2017}{Bryan Hoyle}{Computational Geometry}

\section{Summary of the two methods}

\subsection{hedcuter method}

\subsection{voronoi method}

The voronoi method is rather easy. It iterates a weighted relaxation
based on the density function described in the paper. To compute the
voronoi diagram, it uses Fortune's algorithm. When it's computing the
centroid, it computes the clip lines of the cell and grabs a ``tile''
by using the bounding box of the cell. It then scales the tile to the
desired size (for use in subsampling). Then, it iterates over every
pixel in the tile and uses the clip lines to determine if the pixel is
in the cell. If it is in the cell, it updates both the weight of the
cell and the maximum possible weight of the cell by using the
intensity of the pixel. Then, it computes the radius of the stipple
(if it is set to bound by the cell, it uses the biggest possible
internal radius of the cell, otherwise, it uses the biggest possible
radius that has an edge intersecting the stipple). It uses the ratio
of the weight to the maximum possible weight to scale the radius of
the stipple (the darker the image in the voronoi cell, the larger the
radius). It then sets the center of the cell to the weighted
centroid. It repeats the loop of computing the voronoi diagram then
using weighted relaxation until the average displacement of the
relaxation is very small (meaning that it didn't change much). When it
is done, it just renders the diagram by sampling the cell point for
color (if color mode is enabled, otherwise, it uses a black circle)
and then rendering the stipples as circles of the computed radius.

\section{Comparison of the two methods}

\section{Improvement of hedcuter method}

\bibliographystyle{plain}
\bibliography{report}

\end{document}


